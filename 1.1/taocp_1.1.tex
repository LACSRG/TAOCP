\documentclass{amsart}
\usepackage{taocp}

\title{The Art Of Computer Programming\\Vol 1: Fundamental Algorithms\\Ch 1: Basic Concepts\\Sec 1.1: Algorithms}
\date{Sunday, 2016-07-31}

\begin{document}
\maketitle

% Problem 1
\begin{problem}{[}10{]}
  The text showed how to interchange the values of variables \(m\) and \(n\),
  using the replacement notation, by setting \(t \gets m\), \(m \gets n\),
  \(n \gets t\).  Show how the values of \emph{four} variables
  \((a, b, c, d)\) can be rearranged to \((b, c, d, a)\) by a sequence of
  replacements.  In other words, the new value of \(a\) is to be the original
  value of \(b\), etc.  Try to use the minimum number of replacements.
\end{problem}

\begin{solution}
  \(t \gets a\), \(a \gets b\), \(b \gets c\), \(c \gets d\), \(d \gets t\).
\end{solution}

% Problem 2
\begin{problem} {[}15{]}
  Prove that \(m\) is always greater than \(n\) at the beginning of step E1,
  except possibly the first time this step occurs.
\end{problem}

\begin{solution}
  If \(m \leq n\) at the beginning of step E1 and \(r \neq 0\) after step E1,
  then step E3 sets \(m \gets n\), \(n \gets r\).  By definition, \(n > r\), so
  \(m > n\) every time at the beginning of step E1 after the first time.
\end{solution}

% Problem 3
\begin{problem}{[}20{]}
  Change Algorithm E (for the sake of efficiency) so that all trivial
  replacement operations such as ``\(m \gets n\)" are avoided.  Write this new
  algorithm in the style of Algorithm E, and call it Algorithm F.
\end{problem}

\begin{solution}
\
  \algbegin Algorithm F (Euclid's algorithm without trivial replacements).  Given two
  positive integers \(m\) and \(n\), find their \emph{greatest common divisor},
  that is, the largest positive integer that evenly divides both \(m\) and
  \(n\).
  \algstep F1. [Find remainder 1.] Divide \(m\) by \(n\) and let \(r_1\) be the
  remainder.  (We will have \(0 \leq r_1 < n\).) \\
  \algstep F2. [Is it zero?] If \(r_1 = 0\), the algorithm terminates; \(n\) is the answer. \\
  \algstep F3. [Reduce 1.] Set \(m \gets r_1\). \\
  \algstep F4. [Find remainder 2.] Divide \(n\) by \(m\) and let \(r_2\) be the
  remainder.  (We will have \(0 \leq r_2 < m\).) \\
  \algstep F5. [Is it zero?] If \(r_2 = 0\), the algorithm terminates; \(m\) is the answer. \\
  \algstep F6. [Reduce 2.] Set \(n \gets r_2\), and go back to step F1. \slug
\end{solution}

% Problem 4
\begin{problem}{[}16{]}
  What is the greatest common divisor of 2166 and 6099?
\end{problem}

\begin{solution}
\
\begin{align*}
  &m = 6099, n = 2166, r = 1767 \\
  &m = 2166, n = 1767, r = 399 \\
  &m = 1767, n = 399, r = 171 \\
  &m = 399, n = 171, r = 57 \\
  &m = 171, n = 57, r = 0 \\
  &\textup{gcd}(6099, 2166) = 57
\end{align*}
\end{solution}

% Problem 5
\begin{problem}{[}12{]}
  Show that the ``Procedure for Reading This Set of Books"
\end{problem}

\begin{solution}
\
The procedure fails to have these features:
\begin{enumerate}
  \item Finiteness: Step 18, which is reached upon completion of reading the
    book instructs the reader to start over at the beginning (step 3).
  \item Definiteness: Steps like ``5. Interesting?" and ``14. Tired?"
    are not precisely defined.
  \item Effectiveness: It is not clear that ``12. Work exercises" can be
    completed mechanically with pencil and paper, particularly because some
    exercises have no known solutions and hence may not have any solutions at
    all.
\end{enumerate}
The procedure does have these features:
\begin{enumerate}
  \item Input: The text itself to be read.
  \item Output: Worked exercises.
\end{enumerate}
\end{solution}

% Problem 6
\begin{problem}{[}20{]}
  What is \(T_5\), the average number of times step E1 is performed when
  \(n = 5\)?
\end{problem}

\begin{solution}
\
\begin{align*}
  &(1,5,1) \to (5,1,0) \\
  &(2,5,2) \to (5,2,1) \to (2,1,0) \\
  &(3,5,3) \to (5,3,2) \to (3,2,1) \to (2,1,0) \\
  &(4,5,4) \to (5,4,1) \to (4,1,0) \\
  &(5,5,0) \\
  &T_5 = 13
\end{align*}
\end{solution}

% Problem 7
\begin{problem}{[}M21{]}
  Suppose that \(m\) is known and \(n\) is allowed to range over all positive
  integers; let \(U_m\) be the average number of times that step E1 is executed
  in Algorithm E.  Show that \(U_m\) is well defined.  Is \(U_m\) in any way
  related to \(T_m\)?
\end{problem}

\begin{solution}
\
For \(n > m\), the number of times depends only on \(\mod(n, m)\) and it
suffices to consider only \(n=1,\dots,m\), leading to the same analysis as
above. However, and extra intial step was incurred to swap \(m\) and \(n\),
so we obtain \(U_m = T_m + 1\).
\end{solution}

% Problem 8
\begin{problem}{[}M25{]}
  Give an ``effective" formal algorithm for computing the greatest common
  divisor of positive integers \(m\) and \(n\), by specifying
  \(\theta_j, \phi_j, a_j, b_j\) as in Eqs. (3).  Let the input be represented
  by the string \(a^mb^n\), that is, \(m\) \(a\)'s followed by \(n\) \(b\)'s.
  Try to make your solution as simple as possible.  [\emph{Hint}: Use
  Algorithm E, but instead of division in step E1, set
  \(r \gets |m-n|, n \gets \min(m,n)\).]
\end{problem}

\begin{solution}
  \begin{align*}
    N &= 6 \\
    &j & \theta_j &\to \phi_j & &a_j & &b_j \\
    &0 & ab       &\to c      & &1   & &2 \\
    &1 & cb       &\to bc     & &1   & &0 \\
    &2 & b        &\to b      & &4   & &3 \\
    &3 & a        &\to a      & &5   & &6 \\
    &4 & b        &\to a      & &4   & &5 \\
    &5 & c        &\to b      & &5   & &0
  \end{align*}
\end{solution}

% Problem 9
\begin{problem}{[}M30{]}
  Suppose that \(C_1 = (Q_1, I_1, \Omega_1, f_1)\) and
  \(C_2 = (Q_2, I_2, \Omega_2, f_2)\) are computational methods. For example,
  \(C_1\) might stand for Algorithm E as in Eqs. (2), except that \(m\) and
  \(n\) are restricted in magnitude, and \(C_2\) might stand for a computer
  program implementation of Algorithm E. (Thus \(Q_2\) might be the set of all
  states of the machine, i.e., all ossible configurations of itsmemory and
  registers; \(f_2\) might be the definition of single machine actions; and
  \(I_2\) might bethe initial state, including the program for determining the
  greatest common divisor as well as the values of \(m\) and \(n\).)

  Formulate a set-theoretic definition for the concept ``\(C_2\) is a
  representation of \(C_1\)" or ``\(C_2\) simulates \(C_1\)". This is to mean
  intuitively that any computation sequence of \(C_1\) is mimicked by \(C_2\),
  except that \(C_2\) might take more steps in which to do the computation, and
  it might retain more information in its states. (We thereby obtain a rigorous
  interpretation of the statement, ``Program \(X\) is an implementation of
  Algorithm \(Y\)".)
\end{problem}

\begin{solution}
\begin{definition}
  \(C_2\) simulates \(C_1\) if and only if
  \(\exists g : I_1 \to I_2, h : Q_2 \to Q_1\) such that \(\forall q_0 \in I_1\)
  \(\exists {k_i^{q_0} : \mathbb{N}}_{i=0}^\infty\) such that \(\forall i \geq 0\) we
  have \(h \circ f_2^{k_i^{q_0}} \circ g = f_1^i\).
\end{definition}
\end{solution}

\end{document}
