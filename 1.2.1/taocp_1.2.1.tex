\documentclass{amsart}
\usepackage{taocp}

\title{The Art Of Computer Programming\\
  Vol 1: Fundamental Algorithms\\
  Ch 1: Basic Concepts\\
  Sec 1.2.1: Mathematical Induction}
\date{Saturday, 2016-08-06}

\begin{document}
\maketitle

% Problem 1
\begin{problem}{[}05{]}
  Explain how to modify the idea of proof by mathematical induction, in case we
  want to prove some statement \(P(n)\) for all \emph{nonnegative} integers
  --- that is, for \(n = 0, 1, 2, \dots\) instead of for \(n=1,2,3,\dots\).
\end{problem}

\begin{solution}
  Change step (a) to prove that \(P(0)\) holds. Step (b) remains unchanged.
\end{solution}

% Problem 2
\begin{problem}{[}15{]}
  There must be something wrong with the following proof.  What is it?

  \begin{theorem}
    Let \(a\) be any positive number.  For all positive integers \(n\) we have
    \(a ^{n - 1} = 1\).
  \end{theorem}
  \begin{proof}
    If \(n = 1\), \(a^{n-1}=a^{1-1}=a^0=1\). And by induction, assuming that the
    theorem is true for \(1, 2, \dots, n\), we have
    \begin{equation*}
      a^{(n+1)-1} = a^n = \frac{a^{n-1} \times a^{n-1}}{a^{(n-1)-1}} =
        \frac{1 \time 1}{1} = 1;
    \end{equation*}
    so the theorem is true for \(n+1\) as well.
  \end{proof}
\end{problem}

\begin{solution}
  The proof of the inductive step assumes \(P(1)\), which has not been proven.
  Instead, the base step proved \(P(0)\). Therefore there is no available
  starting point for applying the inductive step.
\end{solution}

% Problem 3
\begin{problem}{[}18{]}
  The following proof by induction seems correct, but for some reason the
  equation for \(n = 6\) gives \(\frac{1}{2} + \frac{1}{6} + \frac{1}{12} +
  \frac{1}{20} + \frac{1}{30} = \frac{5}{6}\) on the left hand side, and
  \(\frac{3}{2} - \frac{1}{6} = \frac{4}{3}\) on the right-hand side. Can you
  find a mistake?
  \begin{theorem}
    \begin{equation*}
      \frac{1}{1 \times 2} + \frac{1}{2 \times 3} + \cdots +
        \frac{1}{(n-1) \times n} = \frac{3}{2} - \frac{1}{n}.
    \end{equation*}
  \end{theorem}
  \begin{proof}
    We use induction on \(n\). For \(n = 1\), clearly \(\frac{3}{2} -
      \frac{1}{n} = \frac{1}{1 \times 2}\); and, assuming that the theorem is
      true for \(n\),
    \begin{align*}
      &\frac{1}{1\times2} + \cdots + \frac{1}{(n - 1) \times n} +
        \frac{1}{n \times (n+1)} \\
      &= \frac{3}{2} - \frac{1}{n} + \frac{1}{n(n+1)} = \frac{3}{2} -
        \frac{1}{n} + \left(\frac{1}{n} - \frac{1}{n+1}\right) = \frac{3}{2} -
        \frac{1}{n+1}.
    \end{align*}
  \end{proof}
\end{problem}

\begin{solution}
  The proof of \(P(1)\) is invalid, since the term \(\frac{1}{1\times2}\)
  corresponds to \(n=2\) rather than to \(n=1\).
\end{solution}

% Problem 4
\begin{problem}{[}20{]}
  Prove that, in addition to Eq. (3), Fibonacci numbers satisfy \(F_n \geq
  \phi^{n-2}\).
\end{problem}

\begin{solution}
  \begin{align*}
    F_1 = 1 &\geq \phi^{-1} \\
    F_2 = 1 &\geq 1 = \phi^0 \\
    F_n = F_{n-1} + F_{n-2} &\geq \phi^{n-3} + \phi^{n-4} = (\phi + 1)\phi^{n-4} = \phi^{n-2} \qed
  \end{align*}
\end{solution}

% Problem 5
\begin{problem}{[}21{]}
  A \emph{prime number} is an integer \(> 1\) that has no exact divisors other
  than 1 and itself. Using this definition and mathematical induction, prove
  that every integer \(> 1\) may be written as a product of one or more prime
  numbers. (A prime number is considered to be the ``product" of a single prime,
  namely itself.)
\end{problem}

\begin{solution}
  Since 2 is prime, it may trivially be written as a product of just one prime,
  namely 2 = 2.

  Let \(n > 2\) be an integer. If \(n\) is prime, then it can trivially be
  written as a product of just one prime, namely \(n = n\). Otherwise,
  \(\exists m,d \in \mathbb{N}\) such that \(n = m * d\). In particular, in
  this case we have \(m > 1\) and \(d > 1\) which implies \(m < n\) and
  \(d < n\). It follows from the inductive hypothesis that \(n=m*d\) can be
  expanded into a product of one or more primes.
\end{solution}

% Problem 6
\begin{problem}{[}20{]}
  Prove that if Eqs. (6) hold just before step E4, they hold afterwards also.
\end{problem}

\begin{solution}
  \begin{align*}
    [a,b,d/a',b',c](a'm + b'n = c) &\equiv & am + bn &= d \\
    [a'-qa,b'-qb,r/a,b,d](am + bn = d) &\equiv & (a'-qa)m + (b'-qb)n &= r \\
    & & a'm + b'n &= q(am + bn) + r \\
    & & &= qd + r \\
    & & &= c \qed
  \end{align*}
\end{solution}

% Problem 7
\begin{problem}{[}23{]}
  Formulate and prove by induction a rule for the sums \(1^2, 2^2 - 1^2, 3^2 -
  2^2 + 1^2, 4^2 - 3^2 + 2^2 - 1^2, 5^2 - 4^2 + 3^2 - 2^2 + 1^2\), etc.
\end{problem}

\begin{solution}
  \begin{theorem}
    \begin{align*}
      \sum_{i=1}^n (-1)^{(n\%2)+1}i^2 &=
        \sum_{i=1}^{\frac{n}{2}} 2(2i)-1 = 2n(n+1) - \frac{n}{2} = 2n^2 + \frac{3}{2}n & &\textup{\(n\) is even}\\
      \sum_{i=1}^n (-1)^{(n\%2)+1}i^2 &=
        \sum_{i=1}^{\frac{n+1}{2}} 2(2i-1)-1 = 2n(n+1) - \frac{3}{2}(n+1) = 2n^2 + \frac{n}{2} - \frac{3}{2} & &\textup{\(n\) is odd} \\
    \end{align*}
  \end{theorem}
\end{solution}

% Problem k
\begin{problem}{[}{]}
\end{problem}

\begin{solution}
\end{solution}

% Problem k
\begin{problem}{[}{]}
\end{problem}

\begin{solution}
\end{solution}

% Problem k
\begin{problem}{[}{]}
\end{problem}

\begin{solution}
\end{solution}

% Problem k
\begin{problem}{[}{]}
\end{problem}

\begin{solution}
\end{solution}

% Problem k
\begin{problem}{[}{]}
\end{problem}

\begin{solution}
\end{solution}

% Problem k
\begin{problem}{[}{]}
\end{problem}

\begin{solution}
\end{solution}

% Problem k
\begin{problem}{[}{]}
\end{problem}

\begin{solution}
\end{solution}

% Problem k
\begin{problem}{[}{]}
\end{problem}

\begin{solution}
\end{solution}

\end{document}
